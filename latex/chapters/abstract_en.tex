\chapter*{Abstract}

Real life datasets usually contain some kind of inherent uncertainty. There have been studies to investigate ways of visualizing this uncertainty. One of those studies was focused on the representation of temporal uncertainty and compared 9 different approaches. A question that was left unanswered by this study though, is when it is advisable to visualize uncertainty at all and when it is better to omit it by using a conventional visualization technique. Thus, we have conducted a user study that compares a conventional technique, without an explicit encoding of uncertainty, and two techniques which present the temporal uncertainty to the user in 4 different types of tasks. Our results show clear tendencies, but could not confirm our hypotheses confidently. To gain further insight in the visualization of temporal uncertainty, we conducted a second user study of exploratory nature. We asked our participants to draw sketches based on given scenarios, which were then analyzed with an open coding approach. This led to the formulation of 12 hypotheses, which could be the focus of future research.