TODO 

.) summary evaluation study. short design...
.) summary of results of evaluation study
.) results are not very conclusive but there is a visible trend -> future work could be a more elaborate and more extensive study

%____________________________________________________________________

Additionally to our quantitative \textit{Evaluation Study} we conducted a exploratory study called the \textit{Drawing Study}. This study has the goal of gaining insights about the intuitiveness of visual encodings for temporal uncertainty. Since the study is of exploratory nature, we did not proof any hypotheses we posed beforehand, but rather generated possible hypotheses from the study results, which could be the focus of future research. During the study we asked the participants to think of possible visualizations for given scenarios and tasks, and sketch them. We collected these drawings and analyzed them with an open coding approach. This means that we looked for similarities and distinctive features and defined categories, in which we could classify the sketches. The respective count of every class is the basis for our analysis.\par \medskip

Through our analysis we generated 12 hypotheses, which can be found in Chapter \ref{ch:discussion}. Most of them are only vaguely supported by our results so far. Because of this it is important to address these in future work and test them through quantitative studies. Since most of the proposed hypotheses regard the intuitiveness of visual encodings in a certain context, we believe that even if they are proven to be true they do not hold much value on their own. The true value lies in the joint insights that can be generated from multiple hypotheses. An example for this is our hypothesis \textbf{H2 Gradient Plots are intuitive representations to support the user in judging a specific probability value of a given point in time.} The knowledge that this visualization technique is intuitive is not valuable on its own, but if we combine it with the study results of \citet{gschwandtner2016visual} that tell us that Gradient Plots are also well fit to support a certain task, we get valuable and deployable insights. \par \medskip

