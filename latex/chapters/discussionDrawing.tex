TODO
.) exploratory study -> find hypotheses 
.) problem task 2 3 merged
task 1:
.) straight forward task -> graph solves this well. seems to be intuitive to read, because people know it (look at demographic of graph drawers)
.) show examples of explicit non-graphs. -> gradient plots for instance. gschwandtner said they are good for this task and they seem to be intuitive!
.) 11 bounded, argue why they draw something like this and research their demographic
.) almost all have time left-right (for all tasks). not a single top-bottom time!

task2:
.) less graphs. maybe become too complex with multiple graphs? (are the 5 graphs from task1 graph-drawers?)
.) people seem to prefer two intervals next to each other to compare
.) people like to have the uncertainty explicit, but there is no go-to technique
.) icons are usually very rough representations and bad for comparison -> none are drawn
.) look at superimposed drawings

task4
.) again not many graphs
.) there is no comparison, but two events at the same time -> same time space -> superposition -> since talking about time space -> clock metaphors
.) since the problem of an overlap probability is so complex, most sketches are bounded
.) if representations were superimposed and did not distinguish themselves by their outline, they were distinguished by color
