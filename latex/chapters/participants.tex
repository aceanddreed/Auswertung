In sum we had 32 participants. Two of them only participated in the \textit{Drawing Study}, but not in the \textit{Evaluation Study}. Hence, we collected 30 and 32 datasets for the two studies respectively. \par \medskip

The participants were chosen people from our friends and family. To make the results as representative as possible, we tried to recruit a heterogeneous group. 20 of our participants are male, while 12 of them are female. Recruiting a heterogeneous group in terms of age was not as easy. We ended up with 24 people under the age of 30 and 8 older ones. \par \medskip

To further given an idea of the participant's demographic we estimated their daily computer usage and knowledge in the field of information visualization and ranked those two factors from in the categories low, average and high. Our participant group shows a rather high density of heavy computer users, with a sum of 17 of them. 10 of them are ranked as average users and only 5 of them use computers seldom. \par \medskip

Since some of the recruited participant are study colleagues of us and also study computer science, they have some knowledge about information visualization. In sum 5 of them completed a bachelor course about this topic and were therefore ranked 'high' by us. 11 participants were ranked average, because they either encounter many data visualizations in their daily work life or studied something technical that involves such visualizations, even though there is no specific information visualization course. The remaining 16 users have no education in this field and work with visualizations only seldom. \par \medskip

An overview of the demographic of the participants is presented in Table \ref{tb:participants}.


\begin{table}[]
	\centering
	\resizebox{\textwidth}{!}{%
	\begin{tabular}{lrll|l|r|r|}
		\cline{2-2} \cline{4-4} \cline{6-7}
		\multicolumn{1}{l|}{\textit{\textbf{Demographic}}} & \multicolumn{1}{r|}{\textbf{Sex}} & \multicolumn{1}{l|}{}             & \textbf{Age}            &                  & \textbf{Computer Usage} & \textbf{InfoVis Knowledge} \\ \hline
		\multicolumn{1}{|l|}{\textit{Male}}                & \multicolumn{1}{r|}{20}           & \multicolumn{1}{l|}{\textit{<30}} & \multicolumn{1}{r|}{24} & \textit{Low}     & 5                       & 16                         \\ \hline
		\multicolumn{1}{|l|}{\textit{Female}}              & \multicolumn{1}{r|}{12}           & \multicolumn{1}{l|}{\textit{>30}} & \multicolumn{1}{r|}{8}  & \textit{Average} & 10                      & 11                         \\ \hline
		& \multicolumn{1}{l}{}              &                                   &                         & \textit{High}    & 17                      & 5                          \\ \cline{5-7} 
	\end{tabular}
}
	\caption{\textit{This table summarizes the demographic of our study participants. A more specific description of it can be found in the \hyperref[ch:participants]{previous section}.}}
	\label{tb:participants}
\end{table}


