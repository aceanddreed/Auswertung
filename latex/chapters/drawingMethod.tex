The \textit{Drawing Study} is designed to be of exploratory nature. This means that the research question it aims to answer is not as concrete as for instance the one of our \textit{Evaluation Study}. The goal is to gain insights into the intuitiveness of visual encodings and to find out how people would visualize temporal uncertainty by themselves. This information could consequently be used in the design of novel visualizations, aimed at expert and especially non-expert users. \par \medskip

To gain these insights, we describe predefined scenarios, which encompass some kind of temporal uncertainty, to our study participants and ask them to draw a visualization sketch that intuitively represents this given scenario. Furthermore, the participants are always provided with a certain task a hypothetical user should be able to efficiently solve given an implementation of the sketched design. \par \medskip

To elicit the desired sketches of temporal visualizations from our study participants, we have to ask the right questions and also have to pay close attention to ask them in the right way. This means that it is imperative to not suggest any possible answers or solution approaches while communicating the task that should be solved, because this would greatly affect their given answers \cite{hullman2016evaluating}. For this reason we try our best to make the given scenario and the task as clear as possible to our participant without suggesting anything that would help in the solution of the task. \par \medskip