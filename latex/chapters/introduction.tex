Datasets containing information retrieved from the real world usually also contain some amount of uncertainty. This uncertainty is often inherent to the data, for instance because certain measurements can never be exact or because some kind of aggregation is already done when acquiring the data. This is also true for temporal data. Sometimes the exact time of an event is not known (e.g., 'time of the big bang'), is given in an inexact way (e.g., 'since a few hours') or is an imprecise prediction of the future (e.g., 'it will take one or two days'). To incorporate these uncertainties into visual representations and make them visible to the user, several approaches have been proposed \cite{kosara2001metaphors, chittaro2001visual, messner2000time, aigner2005planninglines, harris2000information}. \par \medskip

To find out more about the strengths and weaknesses of these techniques and to find out which technique fits certain tasks best, several studies have been conducted. In 2009 Sanyal et al. \cite{sanyal2009user} asked 27 participants to solve four tasks with the help of four commonly used uncertainty visualizations. In 2012 Corell and Gleicher \cite{correll2014error} compared four visual encodings of statistical uncertainty in a user study. In 2012 MacEachren et al. \cite{maceachren2012visual} conducted two studies, which targeted the intuitiveness of visual encodings and their performance in map reading tasks respectively. In 2015 Gschwandtner et al. \cite{gschwandtner2016visual} compared six visual encodings in a comprehensive user study. \par \medskip

To build upon the results of those studies, we are conducting two additional user studies. The first one (referred to as \textit{Drawing Study}) aims to find out more about the intuitiveness of visual encodings. By asking people to draw visualizations for given tasks, we find out how people think about given problems and what kind of representations they think are most appropriate in those situations. The second study (referred to as \textit{Evaluation Study}) is very similar in its design to the one by Gschwandtner et al. \cite{gschwandtner2016visual}. The difference is, that we do not only compare different visual encodings of uncertainty, but also include a representation in the comparison that completely omits the uncertainty of the underlying data. Through this approach we find out in which situations the visualization of uncertainty adds helpful information and in which situations it is only a counterproductive distraction. \par \medskip

In this report we thoroughly describe the design, execution and results of our user studies. Furthermore, we present some of the most relevant related work that has been done, which can be found in \hyperref[ch:related]{Chapter 2}. In \hyperref[ch:method]{Chapter 3} we explain the design of our studies. This chapter is split into three main parts - the first is about the \textit{Evaluation Study}, the second part regards the \textit{Drawing Study} and in the third part the chosen participants and the evaluation approach is addressed. In the following \hyperref[ch:results]{Chapter 4} the results are presented, which are discussed in detail in \hyperref[ch:discussion]{Chapter 5}. We conclude this report with a summary of our approach and its most important findings in \hyperref[ch:conclusion]{Chapter 6}. 